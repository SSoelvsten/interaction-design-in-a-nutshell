%%%%%%%%%%%%%%%%%%%%%%%%%%%
% Set up document         %
%%%%%%%%%%%%%%%%%%%%%%%%%%%
\documentclass[a4, english, twoside]{article}

% Import SSoelvsten's LaTeX preamble
% github.com/SSoelvsten/LaTeX-Preamble_and_Examples
\usepackage{import}
\subimport{../LaTeX-Preamble_and_Examples/preamble/}{preamble_en.tex}

% New settings for this specific document

% Use these for 'definitions' in the theory section
\newtheorem{concept}{Concept}[section]
\newtheorem{framework}{Framework}[section]
\newtheorem{method}{Method}[section]

% Use this for 'examples' in the examples section
\theoremstyle{Example}     % Concepts

% Indexing
% To index something, write \index{keyword}
\usepackage{makeidx}
\usepackage[columns=2]{idxlayout}
\makeindex

%%%%%%%%%%%%%%%%%%%%%%%%%
% Document starts here! %
%%%%%%%%%%%%%%%%%%%%%%%%%

\begin{document}
\settitle{Interaction Design for Computer Scientists}
\addauth{Steffan Sølvsten}{201505832@post.au.dk}{\, au534068}
\maketitle

\begin{abstract}
  \noindent After having to read three different books on Human Computer Interaction, this is an attempt to dispose of the frustrating amount of unecessary information and vague or non-existent definitions in the HCI universe apparent in all text books. This is to be a dense, clearly defined, and small guide to interaction design
\end{abstract}

\newpage
\tableofcontents

\newpage
%%%%%%%%%%%%%%%%%%%%%%%%%%%%%%%%%%%%%%%%%%%%%%%%%%%% 
% Chapter: Introduction
\section{Introduction}
\label{sec:introduction}

This section consists of different real-life examples, which exemplify one or more concepts defined in section \ref{sec:1} and are hence also grouped into equivalent sections.

\newpage
%%%%%%%%%%%%%%%%%%%%%%%%%%%%%%%%%%%%%%%%%%%%%%%%%%%%
% Chapter: Theory and Concepts
\section{Theory and Concepts}
\label{sec:1}

In this section all frameworks, concepts and methods will be defined, compared and criticised. They are simultaneously grouped into the point in development they are to be used. This grouping though makes it seem like a design is a linear process, which it is everything but. The design process should as much as possible follow the idea of agile development in concept \ref{conc:agile}. This means, that no section is ever properly finished, as the designer will keep on going back, but rather the next section is also juggled together with the prior things at the same time.
\begin{concept}[agile development] \label{conc:agile} \index{agile development} 
  
\end{concept}

\begin{framework}[Y-Model] \label{fw:y_model} \index{Y-Model}
  
\end{framework}
\subsection{Initial}
\label{sec:1_initial}


\begin{framework}[PACT] \label{fw:pact} \index{PACT Framework}
  
\end{framework}
\subsection{Data Gathering}
\label{sec:1_data_gathering}

\subsection{Data Analysis}
\label{sec:1_data_analysis}

\subsection{Envisionment}
\label{sec:1_envisionment}

\subsection{Prototyping}
\label{sec:1_prototyping}


\newpage
%%%%%%%%%%%%%%%%%%%%%%%%%%%%%%%%%%%%%%%%%%%%%%%%%%%% 
% Chapter: Examples
\section{Examples}
\label{sec:2}

In this section all frameworks, concepts and methods will be defined, compared and criticised. They are simultaneously grouped into the point in development they are to be used. This grouping though makes it seem like a design is a linear process, which it is everything but. The design process should as much as possible follow the idea of agile development in concept \ref{conc:agile}. This means, that no section is ever properly finished, as the designer will keep on going back, but rather the next section is also juggled together with the prior things at the same time.
\begin{concept}[agile development] \label{conc:agile} \index{agile development} 
  
\end{concept}

\begin{framework}[Y-Model] \label{fw:y_model} \index{Y-Model}
  
\end{framework}
% TODO: Add sections
%       Need to know structure of part 1 first

\newpage
%%%%%%%%%%%%%%%%%%%%%%%%%%%%%%%%%%%%%%%%%%%%%%%%%%%% 
% Chapter: Exercises
\section{Exercises}
\label{sec:3}

In this section all frameworks, concepts and methods will be defined, compared and criticised. They are simultaneously grouped into the point in development they are to be used. This grouping though makes it seem like a design is a linear process, which it is everything but. The design process should as much as possible follow the idea of agile development in concept \ref{conc:agile}. This means, that no section is ever properly finished, as the designer will keep on going back, but rather the next section is also juggled together with the prior things at the same time.
\begin{concept}[agile development] \label{conc:agile} \index{agile development} 
  
\end{concept}

\begin{framework}[Y-Model] \label{fw:y_model} \index{Y-Model}
  
\end{framework}
% TODO: Add sections
%       Need to know structure of part 1 first


\newpage
%%%%%%%%%%%%%%%%%%%%%%%%%%%%%%%%%%%%%%%%%%%%%%%%%%%% 
% References
\begin{minipage}{1.0\textwidth}
  \begin{thebibliography}{9}
  \bibitem{benyon_14}
    Benyon, David: \emph{Designing Interactive Systems}, Pearson, 3rd edition, 2014
  \bibitem{benyon_10}
    Benyon, David: \emph{Designing Interactive Systems}, Pearson, 2nd edition, 2010
  \bibitem{rogers}
    Rogers, Yvonne: \emph{Interaction Design: Beyond human-computer interaction}, Wiley, 3rd edition, 2011
  \bibitem{lim}
    Lim, Youn-Kyung and Tenenberg, Josh: \emph{The Anatomy of Prototypes}, ACM Transactions on Computer-Human Interaction, Vol. 15, 2008
  \end{thebibliography}
  \bibliographystyle{abbrv}
  \bibliography{referencer}
\end{minipage}

\newpage
%%%%%%%%%%%%%%%%%%%%%%%%%%%%%%%%%%%%%%%%%%%%%%%%%%%% 
% Index
\printindex

\end{document}