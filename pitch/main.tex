\documentclass[a4paper, 10pt, twoside, openany]{book}

\usepackage{afterpage}
\newcommand{\todo}{\textcolor{red}{TODO}}

%%%%%%%%%%%%%%%%%%%%%%%%%%%%%%%%%
%      Document Settings        %
%%%%%%%%%%%%%%%%%%%%%%%%%%%%%%%%%
%\makeatletter\@addtoreset{chapter}{part}\makeatother

%%%%%%%%%%%%%%%%%%%%%%%%%%%%%%%%%
%         Page Layout           %
%%%%%%%%%%%%%%%%%%%%%%%%%%%%%%%%%
\usepackage[a4paper]{geometry}
\usepackage{changepage}

\usepackage{lastpage}
\usepackage{fancyhdr}
\pagestyle{fancy}

%Odd pages
\fancyhead[RO]{}
\fancyhead[LO]{\nouppercase{\rightmark}}

%Even pages
\fancyhead[RE]{\nouppercase{\leftmark}}
\fancyhead[LE]{}


%%%%%%%%%%%%%%%%%%%%%%%%%%%%%%%%%
%            Text               %
%%%%%%%%%%%%%%%%%%%%%%%%%%%%%%%%%
% Font
\usepackage[utf8]{inputenc}
\usepackage[T1]{fontenc}
\usepackage{mathpazo}

% Quotes
\usepackage[danish=guillemets]{csquotes}


%%%%%%%%%%%%%%%%%%%%%%%%%%%%%%%%%
% Graphics, figures, and tables %
%%%%%%%%%%%%%%%%%%%%%%%%%%%%%%%%%
\usepackage{booktabs,authblk}
\usepackage{tabularx, longtable}

\usepackage{caption,subcaption}
\usepackage{wrapfig}

\usepackage{graphicx}
\usepackage[dvipsnames]{xcolor}
\usepackage[some,top]{background}

\usepackage{tikz}
\usetikzlibrary{calc}
\usetikzlibrary{positioning}
\usetikzlibrary{shapes}
\usetikzlibrary{decorations.pathmorphing}

%%%%%%%%%%%%%%%%%%%%%%%%%%%%%%%%
%          Hypermedia          %
%%%%%%%%%%%%%%%%%%%%%%%%%%%%%%%%
\usepackage{url}
\usepackage[hidelinks]{hyperref}

%%%%%%%%%%%%%%%%%%%%%%%%%%%%%%%%%
%        Table of Content       %
%%%%%%%%%%%%%%%%%%%%%%%%%%%%%%%%%
\usepackage{minitoc}
\nomtcrule

%%%%%%%%%%%%%%%%%%%%%%%%%%%%%%%%%
%             Index             %
%%%%%%%%%%%%%%%%%%%%%%%%%%%%%%%%%
\usepackage{makeidx}
\usepackage[columns=2]{idxlayout}
\makeindex

%%%%%%%%%%%%%%%%%%%%%%%%%%%%%%%%%
%          Bibliography         %
%%%%%%%%%%%%%%%%%%%%%%%%%%%%%%%%%
\usepackage[
  style=numeric-comp,
  backend=biber
]{biblatex}
\addbibresource{books.bib}
\addbibresource{articles.bib}

%%%%%%%%%%%%%%%%%%%%%%%%%%%%%%%%%
%         Environments          %
%%%%%%%%%%%%%%%%%%%%%%%%%%%%%%%%%
\usepackage{mathtools, amsmath}
\usepackage{amsthm}
\theoremstyle{definition}

\newtheorem{theorem}{Theorem}[chapter]
\newtheorem{definition}[theorem]{Definition}
\newtheorem{remark}[theorem]{Remark}

\newtheorem{principle}[theorem]{Principle}
\newtheorem{concept}[theorem]{Concept}
\newtheorem{framework}[theorem]{Framework}
\newtheorem{method}[theorem]{Method}
\newtheorem{tool}[theorem]{Tool}
\newtheorem{model}[theorem]{Model}



%----------------------------------------------------------------------------------------
% FRONT PAGE SLIDE
%----------------------------------------------------------------------------------------
\title[Rethinking Teaching HCI]{
  Rethinking Teaching Interaction Design: \\ Material and course structure
}

\author{Steffan S\o lvsten} \institute[AU] {
  Aarhus University
  \\ \medskip
  \textit{201505832@post.au.dk}
}
\date{\today}

\begin{document}

\begin{frame}
  \titlepage
\end{frame}

%----------------------------------------------------------------------------------------
% Table of Content
%----------------------------------------------------------------------------------------
\begin{frame}
  \frametitle{Overview}
  \tableofcontents
\end{frame}

%----------------------------------------------------------------------------------------
\section{The Problem}
\begin{frame}
  \frametitle{The Problem}
\end{frame}

%----------------------------------------------------------------------------------------
\subsection{The book}
%----------------------------------------------------------------------------------------
\begin{frame}
  \frametitle{The Problem}
  \Huge{\centerline{The book}}
\end{frame}

%----------------------------------------------------------------------------------------

\begin{frame}
  \textbf{Example 1:} Interaction Design: Beyond human-computer interaction
  
  \medskip
  
  {\tiny \it ``What to evaluate ranges from low-tech prototypes to complete
systems; a particular screen function to the whole workflow; and from aesthetic
design to safety features. For example, developers of a new web browser may want
to know if users find items faster with their product, whereas developers of an
ambient display may be interested in whether it changes people's behaviour.
Government authorities may ask if a computerized system for controlling traffic
lights results in fewer accidents or if a website complies with the standards
required for users with disabilities. Makers of a toy ask if 6-year-olds can
manipulate the controls and whether they are engaged by its furry case, and
whether the toy is safe for them to play with. A company that develops personal,
digital music players may want to know if the size, color, and shape of the
casings are liked by people from different age groups living in different
countries. A software company may want to asses market reaction to its new
homepage design.''}

  \medskip

  \begin{itemize}
  \item After the first two examples the reader either got the point or newer will...
  \end{itemize}
\end{frame}

%----------------------------------------------------------------------------------------

\begin{frame}
  \textbf{Example 2:} Designing Human Interactive Systems
  
  \medskip
  
  {\tiny \it ``One way to conceptualize the main features of a system is to use
a ‘rich picture’. Two examples are shown in Figure 3.2. A rich picture captures
the main conceptual relationships between the main conceptual entities in a
system - a model of the structure of a situation. Peter Checkland (Checkland,
1981; Checkland and Scholes, 1999), who originated the soft systems approach,
also emphasizes focusing on the key transformation of a system. This is the
conceptual model of processing. The principal stakeholders - customers, actors,
system owners - should be identified. The designer should also consider the
perspective from which an activity is being viewed as a system (the
Weltanschauung) and the environment in which the activities take place.
(Checkland proposes the acronym CATWOE - customers, actors, transformation,
Weltanschauung, owners, environment - for these elements of a rich picture.)
Most importantly, the rich picture identifies the issues or concerns of the
stakeholders, thus helping to focus attention on problems or potential design
solutions.''}

  \medskip

  \begin{itemize}
  \item Extremely bad, vague and convoluted way to define a Rich Picture
  \item Quote is from the middle of the description of the Y-model.
  \end{itemize}
\end{frame}

%----------------------------------------------------------------------------------------

\begin{frame}
  \textbf{Example 3:} Designing Human Interactive Systems
  
  \medskip
  
  {\tiny \it ``The key feature of conceptual design is to keep things abstract - focus on the ‘what’
rather than the ‘how’ - and to avoid making assumptions about how functions and infor­
mation will be distributed. There is no clear-cut distinction between conceptual and
physical design, but rather there are degrees of conceptuality.''}
  
  \medskip

  {\tiny \it ``Physical design is concerned with how things are going to work and with detailing
the look and feel of the product. Physical design is about structuring interactions into
logical sequences and about clarifying and presenting the allocation of functions and
knowledge between people and devices. The distinction between conceptual and physi­
cal design is very important.''}

  \begin{itemize}
  \item These two quotes are paragraphs very next to each other on page 51.
  \item There are even bigger contradictions between the literature
  \end{itemize}
\end{frame}

%----------------------------------------------------------------------------------------

\begin{frame}
  We must conclude that the current books used to teach HCI has the following
  properties
  
  \begin{itemize}
  \item Verbosely written
  \item Often goes on irrelevant tangents
  \item Concepts and methods are only vaguely defined
  \item Books contradict each other and/or themselves
  \end{itemize}

  This opinion of the books can be confirmed when interviewing Computer Science students.
\end{frame}

%----------------------------------------------------------------------------------------

\begin{frame}
  Meanwhile what Computer Scientists are good at reading/understanding
  mathematical and computer science books, which all share the following features

  \begin{itemize}
  \item Concise
  \item Stays close to the point currently made
  \item Clear distinction between types of information (definition, examples,
    interluding text)
  \item Everything is defined rigorously
  \item No contradictions
  \end{itemize}

  Which means Computer Scientists have a really hard time coping with the given
  material
\end{frame}

%----------------------------------------------------------------------------------------
\subsection{The course content}
%----------------------------------------------------------------------------------------
\begin{frame}
  \frametitle{The Problem}
  \Huge{\centerline{
      The course content}
  \centerline{and structure}}  
\end{frame}

%----------------------------------------------------------------------------------------

\begin{frame}
  \textbf{Course: Human-Computer Interaction}
  \begin{itemize}
  \item Introduces theory of Interaction Design
  \item Project: Android App
  \end{itemize}

  \medskip
  
  \textbf{Course: Experimental System Development}
  \begin{itemize}
  \item (Re)introduces theory of Interaction Design with some additions
  \item Project 1: Pre-made mini-project
  \item Project 2: Project for client
  \end{itemize}  
\end{frame}

%----------------------------------------------------------------------------------------

\begin{frame}
  Note from this small description, that
  \begin{itemize}
  \item The same theory is taught twice.
  \item A lot of energy is spent on learning to program an Android App.
  \item Hands-on experience of a lot of theory is not possible, as a project
    quickly is locked in a single direction due to technical or time constraints.
  \end{itemize}

\end{frame}

%----------------------------------------------------------------------------------------
\section{A Proposal for a solution}
%----------------------------------------------------------------------------------------
\begin{frame}
  \frametitle{A Proposal for a solution}
  Looking at courses and material that work well for Computer Science students,
  discussing with other students, I (we) have come up with the following suggestions...
\end{frame}

%----------------------------------------------------------------------------------------
\subsection{New book}
%----------------------------------------------------------------------------------------
\begin{frame}
  \frametitle{A Proposal for a solution}
  \Huge{\centerline{New Book}}
\end{frame}

%----------------------------------------------------------------------------------------

\begin{frame}
  Book written similar to the books from the Mathematical Institute
  \begin{itemize}
    \item All definitions, concepts, methods, etc. are clearly labelled,
      numbered, seperated and referenced.
    \item All theory is described in concisely and rigorously
    \item Examples are not inline, but separated
    \item Small concrete exercises are given for all parts of the theory
  \end{itemize}
\end{frame}

%----------------------------------------------------------------------------------------

\begin{frame}
  \textbf{Example} A rigorous, clear, and concise definition

  \medskip
  
  {\small \it 
    ``\textbf{Method 4.13 (Rich Picture): } A diagram of the structure of a
specific situation by depicting the organisational structure, the different
people, and the entities. Relationships and interactions between the different
people and entities are included, together with the concerns of the different
stakeholders. This helps conceptualize the main features of a solution, as it
highlights conflicts and concerns of stakeholders (definition 1.10). Furthermore
as a rich picture is good for investigating and understanding unstructured
domains, then it can be used as part of \emph{understanding} of the
problem domain.''}
\end{frame}

%----------------------------------------------------------------------------------------

\begin{frame}
  This is not just a mere suggestion; I've made progress on exactly this and got
  immensely good feedback on the 'paper prototype'

  \medskip

  \begin{block}{Interaction Design in a Nutshell}
    https://github.com/SSoelvsten/Interaction\_Design\_in\_a\_Nutshell/
  \end{block}
  
  \medskip

  \Huge{\centerline{DEMO}}
\end{frame}


%----------------------------------------------------------------------------------------
\subsection{New course structure}
%----------------------------------------------------------------------------------------
\begin{frame}
  \frametitle{A Proposal for a solution}
  \Huge{\centerline{New course structure}}
\end{frame}

%----------------------------------------------------------------------------------------

\begin{frame}
  \textbf{Course: Human-Computer Interaction}
  \begin{itemize}
  \item Introduces theory of Interaction Design
  \item Small T\O\ Exercises 
  \item 7 - 14 isolated and purely theoretical group hand-ins covering different
    parts of the theory
  \end{itemize}

  \medskip
  
  \textbf{Course: Experimental System Development}
  \begin{itemize}
  \item Project 1: Pre-made mini-project focusing on the interplay of the
    methods and iteration
  \item Project 2: Project for client (bigger than current)
  \end{itemize}
\end{frame}

%----------------------------------------------------------------------------------------

\begin{frame}
  The book exactly attempts to support these new T\O\ exercises and small
  isolated hand-ins.

  \medskip
  
  \textbf{T\O\ Exercise}
  \begin{itemize}
  \item Given a problem description and/or data, ask the students to apply one
    of the methods from the theory.

    \medskip
    
    \textbf{Example:} Given a transcript from an interview create a Flow Model
  \end{itemize}

  \medskip
  
  \textbf{Handins}
  \begin{itemize}
  \item Given a description (and possible data) the students are to apply
    several methods to cover a specific step in the design process.

    \medskip
    
    \textbf{Example:} Given a thorough description of a problem in some domain,
    have the students propose a design and create a paper prototype
  \end{itemize}
\end{frame}

\end{document}
