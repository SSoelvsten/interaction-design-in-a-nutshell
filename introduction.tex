Computer Scientists at Aarhus University tend to have a \emph{love-hate} relationship with all the HCI courses they are forced to take - with the only exception, that they there's no \emph{love} at all. One of the biggest frustrations uttered by most is the lack of a text book, which cuts to the point. All books that have been used, which can be found in the references, are american books, where the authors have been payed per page written. This results in text books of more than 400 pages containing the following

\blockcquote[p. 435]{rogers}{\it What to evaluate ranges from low-tech prototypes to complete systems; a particular screen function to the whole workflow; and from aesthetic design to safety features. For example, developers of a new web browser may want to know if users find items faster with their product, whereas developers of an ambient display may be interested in whether it changes people's behaviour. Government authorities may ask if a computerized system for controlling traffic lights results in fewer accidents or if a website complies with the standards required for users with disabilities. Makers of a toy ask if 6-year-olds can manipulate the controls and whether they are engaged by its furry case, and whether the toy is safe for them to play with. A company that develops personal, digital music players may want to know if the size, color, and shape of the casings are liked by people from different age groups living in different countries. A software company may want to asses market reaction to its new homepage design.}

\noindent If the reader had not already gotten the point by the first example or second, they would certainly not have learned it by the time they reached the end of this monstrosity of a paragraph. Similarly in the 27 pages long paper by Lim and Tenenberg \cite{lim} the same style of writing is evident, where the reader can scan the text and can already reach a conclusion two pages prior to the author.

Simultaneously several methods and concepts are only defined very vaguely or not at all. For example a \emph{rich picture} is in the text book of Benyon only defined with a mere mention to its existence and a reference to a figure with two examples. \cite[p. 51-52]{benyon_14}

It is extremely sad that things like this should result in students dismissing ever looking at exploring and delving into this fascinating field. This document is an attempt to cut to the point, where all concepts, frameworks, methods are clearly defined, compared, and its pros and cons highlighted. We hope this can make the experience of HCI for some students less frustrating, maybe even turn it from an excruviating experience to a delightful one.

This document is divided into three parts: theory, examples, and exercises. It is done this way such that the theory will not drowned in examples and with the exercises we want to provide small and focused samples of data and problems, where the theory can be applied and that lead to discussions. By having the theory this can be used as a quick reference, but with examples and exercises, we hope that this can fully replace the text books and be a basis for a course in interaction design.

\paragraph{Contributors}
I would never have been able to create all of this myself, so I am extremely grateful that this has ended up as a joined effort of so many people. Thanks to everyone who has contributed with smaller og bigger sections, corrections, proof-reading and much more.

\begin{multicols}{2}
  Johannes Ernstsen
\end{multicols}

