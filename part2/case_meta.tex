\chapter{Case: Interaction Design in a Nutshell} \label{chap:case_meta}
\minitoc \newpage

The theory, methods and concepts of this book do not only apply to the design of digital products, but to any kind product that is to be used by humans. As an interesting case study you can consider the very collection of inked sheets of pressed cellulose that you are looking at this very moment. Based on the Y model of section \ref{sec:y_model} this chapter is split into sections for every time the development moved from one activity to another.

\section{Understanding and evaluation of existing material}
As part of common conversations and complaining over lunch the main problems with the current material given in HCI courses were identified to be
\begin{itemize}
  \item A lot of the theory is too fluffy.
  \item Massive amount of theory for trivial ideas.
  \item Examples are interleaved with theory muddying up the theory to learn.
  \item Too many examples for the same point.
\end{itemize}
In the same interviews the following points about the course structure were also highlighted
\begin{itemize}
  \item There are not many exercises given for class.
  \item The few exercises that are given are not efficiently covering the theory.
  \item The projects are extremely long and hence unable to explore a lot of the theory.
\end{itemize}
The understanding from these unstructured interview the key points were casually presented to other students, who all agreed in the problems highlighted.

\section{Conceptual Design and creation of the pitch}
Confident in having identified the root problems in the HCI course literature given, and looking at some of the course literature liked by the same computer scientists, the following main vision was made based on the differences between the two groups of books.
\begin{displayquote}
  \emph{ HCI book that is written exactly like the Linear Algebra and Mathematical Analysis books self-published by the mathematics institute of Aarhus University. This means the concepts and methods will be written in a \emph{definition - proof} format of formal mathematics. Examples are focused on one concepts are put into a seperate part of the book. With the examples are also given larger examples showing the interplay of the methods.}
\end{displayquote}
On the secondary course structure observations the vision was extended with
\begin{displayquote}
  \emph{The book will also contain focused exercises, where prior data and understanding required for the exercises are already provided. It will also contain cases to be solved as one or two week handins, so that students can try solving many problems and explore much more of the theory.}
\end{displayquote}


\subsection{Evaluation with Computer Science students}
In casual conversations over lunch the vision was pitched to other computer science students, who all loved the idea of such course literature existing and being used.

\section{Envisionment through prototyping}
With 

\subsection{Second evaluation with students}


