\section{Scenario based design} \label{sec:scenario_based_design}
Useful in all four stages of the Y-model (section \ref{sec:y_model}), scenarios and personas are one of the most fundamental and popular techniques to designing interactive systems. \cite[p. 62]{benyon14}

\begin{tool}[User Story (Scenario)] \label{tool:user_story} \index{User Story (Scenario)}
  A real-world rendition of the activities, experiences, knowledge etc. of a subject. It is of very high detail, and it can be recorded in many different formats, such as video, text, interview and much more. This is useful to gain an \emph{understanding} of the stakeholders needs and more. \cite[p. 62-63]{benyon_14}
\end{tool}

\begin{tool}[Conceptual Scenario] \label{meth:conceptual_scenario} \index{Conceptual Scenario}
  This is based on Benyon
\end{tool}

\begin{tool}[Concrete Scenario] \label{tool:concrete_scenario} \index{Concrete Scenario}
  A concretization of how a designsolution would work within a specific context and situation as part of the \emph{Conceptual Design}. These to a greater or lesser extend further define interface designs and the different types of and relationship between functions in the system. \cite[p. 64]{benyon14}
\end{tool}
\begin{tool}[Use Case (Scenario)] \label{tool:use_case} \index{Use Case (Scenario)}
  A detailed and meticulous description of an interaction between the user and the system in a highly specific situation. It can take on many forms, such as text, a diagram or pseudocode. Together, as part of the \emph{physical design} (framework \ref{fw:y_model}), a set of use cases give a detailed and complete description of the functionality of the system. \cite[p. 65]{benyon_14}
\end{tool}


A central tool used in the scenario based design is also the \emph{persona}, which can especially enhance both concrete scenarios and use cases and in parts also conceptual scenarios and for revising user stories

\begin{tool}[Persona] \label{tool:persona} \index{Persona}
  A fully fleshed out and concrete characterization of a type of user designed for, including most importantly the persona's background, prerequisites, and goals, but also especially their human characteristics. The more detailed these are the easier they make it for the designer to emphasize with them and as a result design for the actual user. \cite[p. 55]{benyon_14}
\end{tool}

With all four types of scenarios we see all activities of the Y-model are covered. Many user stories show the users needs of the system, abstracted into very few conceptual scenarios defining the vision for the solution. These few then create many concrete scenarios generating concrete solutions, from them many more use cases specifying the exact details of the functions of the system. This relation between the stories, the design process and the Y-model is visualized in figure \ref{fig:scenarios}. \cite[p. 66, 196]{benyon14}

\begin{figure}
  \centering
  TODO : Recreate the figure from Benyon
  \caption{The four different types of scenarios \cite[p. 67]{benyon_14}}
  \label{fig:scenarios}
\end{figure}


\begin{definition}[Scenario Corpus] \label{def:scenario_corpus} \index{Scenario Corpus}
  A set of user stories picked out of the many collected, which together portray a cohesive and exhaustive description of the problem domain. This will highlight the different \emph{dimensions} over which the system to be designed spans, such as the functions, content, aesthetics and more. A scenario corpus should attempt to cover \cite[p. 67-68, 197]{benyon14}
  \begin{itemize}
    \item All types of interactions
    \item Design issues deemed important for the project
    \item Areas with yet unclear requirements
    \item Any safety-critical (Section \ref{sec:pact}) aspect
  \end{itemize}
  To create the corpus the designers should first annotate all scenarios with information, which then can be used to sort and merge the many scenarios. \cite[p. 198]{benyon14}   
\end{definition}

While the PACT framework of section \ref{sec:pact} is used to critique the scenarios, the fast growing amount of scenarios needs to be managed. To help collaboration and conveying the point of the scenarios they can be annotated with a description of the personas included, the activities covered, and the key points of the scenario among other things. These descriptions can be further annotated with meta data, such as the author of and their rationale for the scenario, its changehistory, and the domains to which it generalizes. Furthermore how the scenarios are related to eachother, such as what scenario spawned another, can be very valuable \cite[p. 70,72]{benyon14}

\begin{definition}[Endnote] \label{def:endnote} \index{Endnote}
 \cite[p. 70]{benyon14}
\end{definition}
\begin{method}[Trade-offs and claims analysis] \label{meth:trade-offs_and_claims} \index{Trade-offs and claims analysis}
 \cite[p. 70]{benyon_14} 
\end{method}
\begin{method}[Object-Action analysis] \label{meth:object-action_analysis} \index{Object-Action analysis}
  
\end{method}

Together all of this can be used to derive the conceptual model and design language of section \ref{sec:conceptual_model}, and from method \ref{meth:object-action_analysis} can tool \ref{tool:conceptual_model} be derived. \cite[p. 67]{benyon14}
