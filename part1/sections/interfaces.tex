\section{\todo : Interfaces}
With different technology comes many different interfaces

\begin{definition}[Interface] \label{def:interface} \index{Interface}
  The physical, visual and conceptual aspects that mediate the interaction of users with the system. \cite[p. 256]{benyon14}
\end{definition}


Where as classical user interfaces have primarily been focusing on a single user interacting at a time, with the move towards Ubiquitous Computing (\todo ) more focus has been put into interfaces that supports use by multiple users. \cite[p. 157-158]{rogers}

\begin{definition}[Shareable Interface] \label{def:shareable_interface} \index{Shareable Interface}
  
\end{definition}

An important aspect of different interfaces and their elements are their \emph{affordances}. Using them can make it much easier for the user to quickly learn the system, while misusing or ignoring them can lead to the user getting confused.

\begin{definition}[Affordance] \label{def:affordance} \index{Affordance}
  An attribute of an object, which signals how to use it. We distinguish between two types of affordances.
  \begin{itemize}
    \item Real affordances: A property of physical objects, where the affordance is obvious and does not need to be learned.
    \item Perceived affordances: A property of non-physical objects, i.e. virtual interfaces. These do not have the trivial real affordance, meaning they in other words are learned conventions.
    \end{itemize}
    Just as a physical object with easily readable affordances makes it obvious how the object should be used, an interface should attempt to do the same. \cite[p. 30]{rogers}
\end{definition}

This section is a collection of the many different kinds of interfaces. These interfaces are grouped together, though not too much emphasis should be put into the meaning of the groupings, as they are only meant for the reader to have a better overview. It should also be noted, that a system can be classified as more than one of the interface types at the same time.

\subsection{Keybased interfaces}

\begin{definition}[Command-line Interface] \label{def:command-line_interface} \index{Interface!Command-line}
  
\end{definition}
\begin{definition}[WIMP Interface] \label{def:WIMP} \index{Interface!WIMP}
  
\end{definition}
\begin{definition}[Graphical User Interface] \label{def:graphical_user_interface} \index{Graphical User Interface} \index{GUI}
  Normally referred to as \emph{GUI}
\end{definition}
\begin{definition}[Web Based Interface] \label{def:web_based_interface} \index{Interface!Web based}
  
\end{definition}

\subsection{\emph{Natural} interfaces}
The use of \emph{natural} and \emph{intuitive} to describe an interface is highly discouraged as it will dilude the meaning of these two words. The reason this group of interfaces is named so anyways, is to highlight that these interfaces attempt to provide an interaction that is closer to how humans are used tointeract with the physical world.

\begin{definition}[Pen Based Interface] \label{def:pen_based_interface} \index{Interface!Pen based}
  
\end{definition}
\begin{definition}[Touch Based Interface] \label{def:touch_based} \index{Interface!Touch based} \index{TUI}
  Also known as \emph{TUI}
\end{definition}
\begin{definition}[Gesture Based Interface] \label{def:gesture_based_interface} \index{Interface!Gesture based}
  
\end{definition}
\begin{definition}[Tangible Interface] \label{def:tangible_interface} \index{Interface!Tangible}
  A sensor-based interaction, where the manipulation physical objects is coupled together with manipulating digital. The sensors can be embedded in the physical objects manipulated or the surrounding environment.

  Normally a tangible interface uses several physical objects, where the interaction can be designed using the objects affordances in mind to help learning to use the system. The use of physical objects and their relative positioning and state lends itself to many more intricate data representations and interactions than other media. Furthermore a tangible interface lends itself well for a shareable interface (Definition \ref{def:shareable_interface}) \cite[p. 205-206]{rogers}
\end{definition}

\subsection{Digital Realities}

\begin{definition}[Virtual Reality] \label{def:virtual_reality} \index{Virtual Reality} \index{Reality!Virtual}
  
\end{definition}
\begin{definition}[Augmented Reality] \label{def:augmented_reality} \index{Augmented Reality}
  
\end{definition}

\subsection{Ubiquitous User Interfaces} \index{Ubiquitous!User Interfaces} \index{UUI}

\begin{definition}[Ambient Interface] \label{def:ambient} \index{Ambient Interface}
  
\end{definition}
\begin{definition}[Wearable Interface] \label{def:wearable_interface} \index{Interface!Wearable}
  
\end{definition}
\begin{definition}[Appliance Interface] \label{def:appliance_interface} \index{Appliance Interface}
  
\end{definition}

\subsection{\todo : To be categorized}

\begin{definition}[Mobile Interface] \label{def:mobile_interface} \index{Mobile Interface}
  
\end{definition}
\begin{definition}[Robot Interface] \label{def:robot_interface} \index{Robot Interface}
  
\end{definition}
\begin{definition}[Shape changing Interface] \label{def:shape_changing_interface} \index{Interface!Shape changing}
  
\end{definition}
