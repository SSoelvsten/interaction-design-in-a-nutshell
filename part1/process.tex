\chapter{Process of human-centred design} \label{chap:process}
\minitoc \newpage

The key for modern design of human-computer interaction is to take the human, their needs and abilities, as a starting point. A designsolution unusable by the intended users is a failure the designer is responsible for. Hence, the core part of successful human-centred design is to constantly evaluate everything, preferably with the stakeholders.

But why even bother going through all the hazzle and ressources and instead just skip everything contained in this book and just make a product? By having the human central in the design will create a product more likely to be widely adopted and used longer, which likely gives a big return on investment. Furthermore several pitfalls in terms of safety and ethics can be circumvented, if the design is human-centred. Furthermore the design of technologies need to accomodate and support human values. \cite[p. 20-22]{benyon14}

\section{Y-model} \label{sec:y_model} \index{Y-Model}
The Y-model is a characterization of the overall design process, dividing the designer's work into the following four activities
\begin{itemize}
\item \emph{Evaluation}: \index{Evaluation}
  Everything produced by the designer, be it understanding of the users needs and context, abstract design ideas or concrete prototypes, has to be evaluated with the end users. The type of evaluation varies heavily based on what is evaluated and the focus of the designer.
\item \emph{Understanding}: \index{Understanding}
  Research of stakeholders (definition \ref{def:stakeholder}) and their activities and the contexts within the problem domain. On an abstract level the designer wants to identify the stakeholders goals, requirements, wishes and needs, together with the restrictions on technologies and people due to the surrounding context.
\item \emph{Design}:
  Any part of the design, which are not immediately resulting in a physical product, but rather on a more abstract level specifies the product.
  \begin{itemize}
  \item \emph{Conceptual Design}: \index{Conceptual Design}
    On an abtract level the overall purpose of the system. This includes what functionality, structure, and information is provided by the system designed. These are specified together with finding a conceptual model (concept \ref{conc:conceptual_model}) to clearly communicate the design to the user. This can be manifested in a specification and vision of the system or through other ways, but should be as independent of implementations as possible.
    
  \item \emph{Physical Design}: \index{Physical Design}
    Going from the abstract to the concrete, the designer specifies the sequence of interactions, the presentation of information and functions, and the physical feel of the system. It consists of the following three components
    \begin{itemize}
    \item \emph{Operational Design}: \index{Operational Design}
      The way functions and information is structured relative to each other throughout the system
      
    \item \emph{Representational Design}: \index{Representational Design}
      The overall feel and look of the design
      
    \item \emph{Interaction Design}: \index{Interaction Design}
      The structure and sequence of the interactions between humans and the system.
      
    \end{itemize}
  \end{itemize}

\item \emph{Envisionment}: \index{Envisionment}
  Making the design ideas real and bringing them physically into the world, the designers create everything from simple sketches to full blown high-fidelity prototypes (definition \ref{def:hi-fi_prototype}). With these in hand the designers can concretely evaluate their ideas with eachother and with the users.
  
\end{itemize}
As visualized in figure \ref{fig:y_model} the design process is non-linear and evaluation is central to the process. The designer may start in any of the four activities and everything by the designer is evaluated before moving onto the next activity. \cite[p. 49-54, 188, 206]{benyon_14}

\begin{figure}[ht!]
  \centering
  \begin{tikzpicture}
    \node[cloud, draw, aspect=2] (evaluation) {\bf Evaluation};
    \node[cloud, draw=none, aspect=3] (top_guide) [above=1cm of evaluation] {};
    \node[cloud, draw, aspect=3] (envisionment) [left=0.1cm of top_guide] {\bf Envisionment};
    \node[cloud, draw, aspect=3] (understanding) [right=0.1cm of top_guide] {\bf Understanding};
    \node[cloud, draw, aspect=3, align=left] (design) [below =1cm of evaluation] {\bf Design \\ \\};
    \node[cloud, draw, fill=white, aspect=2, align=right] (conceptual) [below left =-1.8cm of design] {\bf Conceptual \\ \bf Design};
    \node[cloud, draw, fill=white, aspect=2, align=left] (physical) [below right =-1.8cm of design] {\bf Physical \\ \bf Design};
    \path[<->, line width=0.6mm]
        (evaluation) edge (envisionment)
                     edge (understanding)
                     edge (design)
    ;
  \end{tikzpicture}
  \caption[The Y-model]{The Y-model visualized \cite[p. 49]{benyon14}}
  \label{fig:y_model}
\end{figure}

\section{PACT} \label{sec:pact}

\subsection{People}

\subsection{Activities}

\subsection{Context}

\subsection{Technologies}
\section{Scenario based design} \label{sec:scenario_based_design}
Useful in all four stages of the Y-model (section \ref{sec:y_model}), scenarios and personas are one of the most fundamental and popular techniques to designing interactive systems. \cite[p. 62]{benyon_14}

\begin{tool}[User Story (Scenario)] \label{tool:user_story} \index{User Story (Scenario)}
  A real-world rendition of the activities, experiences, knowledge etc. of a subject. It is of very high detail, and it can be recorded in many different formats, such as video, text, interview and much more. This is useful to gain an \emph{understanding} of the stakeholders needs and more. \cite[p. 62-63]{benyon_14}
\end{tool}

\begin{tool}[Conceptual Scenario] \label{meth:conceptual_scenario} \index{Conceptual Scenario}
  This is based on Benyon
\end{tool}

\begin{tool}[Concrete Scenario] \label{tool:concrete_scenario} \index{Concrete Scenario}
  A concretization of how a designsolution would work within a specific context and situation as part of the \emph{Conceptual Design}. These to a greater or lesser extend further define interface designs and the different types of and relationship between functions in the system. \cite[p. 64]{benyon14}
\end{tool}
\begin{tool}[Use Case (Scenario)] \label{tool:use_case} \index{Use Case (Scenario)}
  A detailed and meticulous description of an interaction between the user and the system in a highly specific situation. It can take on many forms, such as text, a diagram or pseudocode. Together, as part of the \emph{physical design} (framework \ref{fw:y_model}), a set of use cases give a detailed and complete description of the functionality of the system. \cite[p. 65]{benyon_14}
\end{tool}


A central tool used in the scenario based design is also the \emph{persona}, which can especially enhance both concrete scenarios and use cases and in parts also conceptual scenarios and for revising user stories

\begin{tool}[Persona] \label{tool:persona} \index{Persona}
  A fully fleshed out and concrete characterization of a type of user designed for, including most importantly the persona's background, prerequisites, and goals, but also especially their human characteristics. The more detailed these are the easier they make it for the designer to emphasize with them and as a result design for the actual user. \cite[p. 55]{benyon_14}
\end{tool}

With all four types of scenarios we see all activities of the Y-model are covered. Many user stories show the users needs of the system, abstracted into very few conceptual scenarios defining the vision for the solution. These few then create many concrete scenarios generating concrete solutions, from them many more use cases specifying the exact details of the solution. This relation between the stories, the design process and the Y-model is visualized in figure \ref{fig:scenarios}.

\begin{figure}
  \centering
  TODO : Recreate the figure from Benyon
  \caption{The four different types of scenarios \cite[p. 67]{benyon_14}}
  \label{fig:scenarios}
\end{figure}


\begin{definition}[Scenario Corpus] \label{def:scenario_corpus} \index{Scenario Corpus}
  A set of user stories picked out of the many collected, which together portray a cohesive and exhaustive description of the problem domain. This will highlight the different \emph{dimensions} over which the system to be designed spans, such as the functions, content, aesthetics and more. A scenario corpus should attempt to cover \cite[p. 67-68, 197]{benyon14}
  \begin{itemize}
    \item All types of interactions
    \item Design issues deemed important for the project
    \item Areas with yet unclear requirements
    \item Any safety-critical (Section \ref{sec:pact}) aspect
  \end{itemize}
  To create the corpus the designers should first annotate all scenarios with information, which then can be used to sort and merge the many scenarios. \cite[p. 198]{benyon14}   
\end{definition}

