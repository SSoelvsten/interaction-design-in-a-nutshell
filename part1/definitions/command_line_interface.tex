\begin{definition}[Command-line Interface] \label{def:command-line_interface} \index{Interface!Command-line}
  A key-based interaction, where the user executes commands or functions through writing commands and its arguments, pre-defined keys or key-combinations. While this comes short in respect to being easy to learn for the user, when first learned they allow the user rapid and precise interactions. Because of this, they can still be seen in software or parts of software designed for expert users. Being purely text-based this can also be used to allow visually impaired users to interact with the system.
  \cite[p. 159]{rogers}
\end{definition}

\begin{remark}
  While more specific, user chosen or high-frequency command names may help users learn to use the system, the most important thing is to be consistent. Abbreviations should always be the first letter of the operation. 
  \cite[p. 160]{rogers}
\end{remark}