\begin{concept}[Mental Model] \label{conc:mental_model} \index{Mental!Model}
  As users read about, observe and interact with a system they develop a \emph{mental model} of how the system operates and the relationsship between its parts. Some key properties of mental models are that they are
  \begin{itemize}
    \item incomplete and finer in detail on some parts compared to others.
    \item unstable, as people easily forget details.
    \item not bounded to the specific system, as users can confuse it with the mental model of similar systems or reapply the current on other systems.
    \item unscientific and sometimes with \emph{superstitious} elements.
    \item parsimonious, i.e. users gladly do more physical actions to minimize mental efforts.
    \item possibly based on inappropriate analogies.
  \end{itemize}
  Furthermore users can in a \emph{mental simulation} run their mental model, predicting with limited accuracy the outcome of a set of actions. \cite[p. 31-32]{benyon14} \cite[p. 86-88]{rogers} \index{Mental!Simulation}

  If a user does not have a good mental model of a system, then most interactions performed by them will be memorized and they will be unable to recover from failure and errors - not understanding what went wrong. \cite[p. 31-32]{benyon14}
\end{concept}