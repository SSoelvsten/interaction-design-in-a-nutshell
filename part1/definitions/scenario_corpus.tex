\begin{definition}[Scenario Corpus] \label{def:scenario_corpus} \index{Scenario Corpus}
  A set of user stories picked out of the many collected, which together portray a cohesive and exhaustive description of the problem domain. This will highlight the different \emph{dimensions} over which the system to be designed spans, such as the functions, content, aesthetics and more. A scenario corpus should attempt to cover \cite[p. 67-68, 197]{benyon14}
  \begin{itemize}
    \item All types of interactions
    \item Design issues deemed important for the project
    \item Areas with yet unclear requirements
    \item Any safety-critical (Section \ref{sec:pact}) aspect
  \end{itemize}
  To create the corpus the designers should first annotate all scenarios with information, which then can be used to sort and merge the many scenarios. \cite[p. 198]{benyon14}   
\end{definition}